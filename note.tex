% ============================================================================
%
% 物理学まとめノート
%
%
%
%
% ============================================================================
\documentclass[a4j]{jsbook}
\usepackage{amsmath, amsthm, amssymb, ascmac}
\usepackage{braket}



% ============================================================================
%
% マクロ定義
%
%
% 独自に作成したマクロ
% ============================================================================

% ============================================================================
%
% 複素数
%
% 要件等 : Im, Reの後ろの空白の幅
%
% ============================================================================
\def\Re{{\rm Re} \  }
\def\Im{{\rm Im} \  }


% ============================================================================
%
% ベクトル場 微分演算子
%
%
% ============================================================================
\def\grad{ \mathrm{grad} \  }
\def\div{ \mathrm{div} \  }
\def\rot{ \mathrm{rot} \  }




\setlength{\textwidth}{\fullwidth}
\setlength{\evensidemargin}{\oddsidemargin}

\begin{document}
% ============================================================================
%
% 数学
%
% ============================================================================

% ============================================================================
%
% 解析学
%
% ============================================================================
\chapter{解析学}
% ============================================================================
%
% 物理学まとめノート
%
%
%
%
% ============================================================================
\section{Limit (極限)}

% ============================================================================
%
% 物理学まとめノート
%
%
%
%
% ============================================================================
\section{Derivative (微分)}

% ============================================================================
%
% 物理学まとめノート
%
%
%
%
% ============================================================================
\section{Integral (積分)}




% ============================================================================
%
% ベクトル解析
%
% ============================================================================
\chapter{ベクトル解析}
% ============================================================================
%
% 物理学まとめノート
%
%
%
%
% ============================================================================
\section{Coordinate system (座標系)}


\subsection{Cartesian coordinate system (デカルト座標)}


\subsection{Polar coordinate system (極座標)}
\begin{align}
  x & = r\cos \theta \\
  y & = r\sin \theta
\end{align}


\subsection{Parabolic coordinates (放物線座標)}
Two-dimensional parabolic coordinates $ (\sigma ,\tau ) $ are defined by the equations
\begin{align}
  x & = \sigma \tau \\
  y & = \frac{1}{2} \left( \tau^{2} - \sigma^{2} \right)
\end{align}

The curves of constant $ \sigma  $ form confocal parabolae
\begin{equation}
  2y = \frac{x^{2}}{\sigma^{2}} - \sigma^{2}
\end{equation}



\subsection{Cylindrical coordinate system (円柱座標)}
\begin{align}
  x & = \rho \cos \varphi  \\
  y & = \rho \sin \varphi \\
  z & = z
\end{align}

\begin{equation}
  d \bf{r} = d \rho \bf{e_{\rho}} + \rho d \varphi \bf{e_{\varphi}} + dz \bf{e_{z}}
\end{equation}

\begin{equation}
  d \bf{V} = \rho d \rho  d \varphi dz
\end{equation}



\subsection{Parabolic cylindrical coordinates (放物線柱座標)}
\begin{align}
  x & = \sigma \tau \\
  y & = \frac{1}{2} \left( \tau^{2} - \sigma^{2} \right) \\
  z & = z
\end{align}

% ============================================================================
%
% 物理学まとめノート
%
%
%
%
% ============================================================================
\section{Gradient (勾配)}

\begin{align}
  \nabla f
      & = \frac{\partial f}{\partial x}\mathbf{e}_x + \frac{\partial f}{\partial y}\mathbf{e}_y + \frac{\partial f}{\partial z}\mathbf{e}_z  \\
      & = \frac{\partial f}{\partial \rho}\mathbf{e}_\rho + \frac{1}{\rho}\frac{\partial f}{\partial \theta}\mathbf{e}_\theta + \frac{\partial f}{\partial z}\mathbf{e}_z \\
      & = \frac{\partial f}{\partial r}\mathbf{e}_r + \frac{1}{r}\frac{\partial f}{\partial \theta}\mathbf{e}_\theta + \frac{1}{r \sin\theta}\frac{\partial f}{\partial \phi}\mathbf{e}_\phi
\end{align}

% ============================================================================
%
% 物理学まとめノート
%
%
%
%
% ============================================================================
\section{Divergence (発散)}

\begin{align}
  \nabla \cdot \mathbf{A}
      & = {\partial A_x \over \partial x} + {\partial A_y \over \partial y} + {\partial A_z \over \partial z} \\
      & = {1 \over \rho}{\partial \left( \rho A_\rho \right) \over \partial \rho} + {1 \over \rho}{\partial A_\phi \over \partial \phi} + {\partial A_z \over \partial z} \\
      & = {1 \over r^2}{\partial \left( r^2 A_r \right) \over \partial r} + {1 \over r\sin\theta}{\partial \over \partial \theta} \left( A_\theta\sin\theta \right) + {1 \over r\sin\theta}{\partial A_\phi \over \partial \phi}
\end{align}

% ============================================================================
%
% 物理学まとめノート
%
%
%
%
% ============================================================================
\section{Curl (回転)}

\begin{eqnarray*}
\nabla \times \mathbf{A}  & = &   \left({\partial A_z \over \partial y} - {\partial A_y \over \partial z}\right) \mathbf{\hat x} + \left({\partial A_x \over \partial z} - {\partial A_z \over \partial x}\right) \mathbf{\hat y} + \left({\partial A_y \over \partial x} - {\partial A_x \over \partial y}\right) \mathbf{\hat z} \\
                  & = & \left({1 \over \rho}{\partial A_z \over \partial \phi} - {\partial A_\phi \over \partial z}\right) {\hat \rho} + \left({\partial A_\rho \over \partial z} - {\partial A_z \over \partial \rho}\right) {\hat \phi} + {1 \over \rho}\left({\partial \left( \rho A_\phi \right) \over \partial \rho} - {\partial A_\rho \over \partial \phi}\right) {\hat z} \\
                  & = & {1 \over r\sin\theta}\left({\partial \over \partial \theta} \left( A_\phi\sin\theta \right) - {\partial A_\theta \over \partial \phi}\right) {\hat r} + {1 \over r}\left({1 \over \sin\theta}{\partial A_r \over \partial \phi} - {\partial \over \partial r} \left( r A_\phi \right) \right) {\hat \theta} \\
                  & &   + {1 \over r}\left({\partial \over \partial r} \left( r A_\theta \right) - {\partial A_r \over \partial \theta}\right) {\hat \phi}
\end{eqnarray*}

% ============================================================================
%
% 物理学まとめノート
%
%
%
%
% ============================================================================
\section{Del (ナブラ演算子)}


\begin{eqnarray*}
\rm curl \ \rm grad f       & = & \mathbf{0} \\
\rm div \ \rm curl \mathbf{A} & = & \mathbf{0} \\
\rm curl \rm curl \mathbf{A}    & = & \nabla (\nabla \cdot \mathbf{A}) - \nabla^2 \mathbf{A}
\end{eqnarray*}

% ============================================================================
%
% 物理学まとめノート
%
%
%
%
% ============================================================================
\section{Laplace operator (ラプラス演算子)}

\begin{eqnarray*}
\Delta f  & = & \frac{\partial^2 f}{\partial x^2} + \frac{\partial^2 f}{\partial y^2} + \frac{\partial^2 f}{\partial z^2} \\
      & = & {1 \over \rho} {\partial \over \partial \rho} \left( \rho {\partial f \over \partial \rho} \right) + {1 \over \rho^2} {\partial^2 f \over \partial \theta^2} + {\partial^2 f \over \partial z^2 } \\
      & = & {1 \over r^2} {\partial \over \partial r} \left( r^2 {\partial f \over \partial r} \right) + {1 \over r^2 \sin \theta} {\partial \over \partial \theta} \left( \sin \theta {\partial f \over \partial \theta} \right) + {1 \over r^2 \sin^2 \theta} {\partial^2 f \over \partial \phi^2}
\end{eqnarray*}




% ============================================================================
%
% 特殊関数
%
% ============================================================================
\chapter{特殊関数}
% ============================================================================
%
% 物理学まとめノート
%
%
%
%
% ============================================================================
\section{Gamma function (ガンマ関数)}


\subsection{Definition}

\subsubsection{Main definition}
\begin{equation}
  \Gamma \left( z \right) = \int^{\infty }_{0} t^{z-1} e^{-t} dt
\end{equation}

\begin{align}
  \Gamma \left( z + 1 \right)
      & = z \Gamma \left( z \right) \\
  \Gamma \left( n \right)
      & = \left( n - 1 \right) !
\end{align}



\subsubsection{Alternative definitions}
\begin{align}
  \Gamma \left( z \right)
      & = \lim_{ n \to \infty} \frac{ n! n^z } { z \left( z+1 \right) \cdots \left( z + n \right) } \\
      & = \frac{ 1 }{ z } \prod_{n=1}^{\infty } \frac{ {\left( 1 + 1/n \right)}^{z} }{ 1 + z/n } \\
  \Gamma \left( z \right)
      & = \frac{ e^{ -\gamma z } }{ z } \prod_{n=1}^{\infty } { \left( 1 + \frac{z}{n} \right) }^{-1} e^{z/n}
\end{align}

where $ \gamma \approx 0.577216 $ is the Euler-Mascheroni constant.
\begin{align}
  \gamma
    & = \lim_{ n \to \infty } \left( \sum_{k=1}^{n} \frac{1}{k} - \ln (n) \right) \\
    & = \int_{1}^{\infty } \left( \frac{1}{?x?} - \frac{1}{x} \right) dx
\end{align}


\subsection{Properties}




% ============================================================================
%
% 物理学
%
% ============================================================================

% ============================================================================
%
% 力学
%
% ============================================================================
\chapter{力学}
\section{剛体}


\subsection{慣性モーメント}
\begin{align}
  I
    & = \sum m_i {r_i}^2 \\
    & = \int_{V} r^2 dm \\
    & = \int_{V} r^2 \rho \left( r \right) dV\vartheta
\end{align}

ここで、 $ m_i $ は微小部分の質量、 $ r_i $ は回転軸からの距離。




% ============================================================================
%
% 解析力学
%
% ============================================================================
\chapter{解析力学}
% ============================================================================
%
% ラグランジュ力学
%
% ============================================================================
\section{Lagrangian mechanics (ラグランジュ力学)}


\subsection{Virtual work principle (仮想仕事の原理)}
\begin{equation}
  {\delta}' W = \sum^{N}_{i=1} {\bf F}_i \cdot \delta {\bf r}_i = 0
\end{equation}

\subsection{D'Alembert's principle (ダランベールの原理)}
\begin{equation}
  {\bf K}_i - m_i \ddot{\bf{r}} = 0
\end{equation}

\subsection{(ラグランジュの変分方程式)}
\[ \sum^{N}_{i=1} \left( {\bf F}_i + {\bf S}_i - m_i \ddot{\bf r}_i \right) \cdot \delta {\bf r}_i \]
\[ \sum^{N}_{i=1} \left( {\bf F}_i - m_i \ddot{\bf r}_i \right) \cdot \delta {\bf r}_i \]


\subsection{(第一種ラグランジュ運動方程式)}
\begin{equation}
  m_i \ddot{x}_i = F_i + \sum^{h}_{l=1} {\lambda}_l \frac{ \partial f_l }{ \partial x_i } ,\ \ \  i = 1,2,\cdots ,n
\end{equation}
$ \lambda_i $ are undetermined Lagrange multipliers.


\subsection{Lagrange's equations (第二種ラグランジュの運動方程式)}
\begin{equation}
  \frac{ d }{ dt } \left( \frac{ \partial L }{ \partial \dot{q}_i } \right) - \frac{ \partial L }{ \partial q_i } = 0 , \ \ \  i = 1,2,\cdots , f
\end{equation}


\subsection{Examples}
\begin{align}
  L & = T (\dot q) - V(q) \\
  L & = \frac{ 1 }{ 2 } m \dot{{\bf{r}}^2} - e \phi + e \dot{ \bf r } \cdot \bf{A} \\
\end{align}

% ============================================================================
%
% ハミルトニアン力学
%
% ============================================================================
\section{Hamiltonian mechanics (ハミルトニアン力学)}


\subsection{generalized momenta (一般運動量)}
\begin{equation}
  p_i = \frac{ \partial L }{ \partial \dot{q}_i }
\end{equation}


\subsection{Hamiltonian (ハミルトニアン)}
\begin{align}
  H & = \sum_i p_i \dot{q}_i - L \\
  dH
      & = \sum_i \left[ \left( F_i - {\dot p}_i \right) d q_i + {\dot q_i} d p_i \right] - \frac{ \partial L }{\partial t} d t \\
      & = \sum_i \left[ \frac{ \partial H }{ \partial q_i } d q_i + \frac{ \partial H }{ \partial p_i } d p_i \right] + \frac{ \partial H }{ \partial t } d t
\end{align}


\subsection{Hamilton's canonical equations (ハミルトンの正準方程式)}
\begin{align}
  \frac{ \partial H }{ \partial q_i }
      & = F_i - \dot{p}_i \\
  \frac{ \partial H }{ \partial p_i }
      & = \dot{q}_i \\
  \frac{ \partial H }{ \partial t }
      & = - \frac{ \partial L }{ \partial t }
\end{align}


\subsection{Poisson bracket (ポアソンの括弧式)}
\begin{align}
  \left\{ X , H \right\}
      & = \sum_i \left( \frac{\partial X}{\partial q_i}\frac{\partial H}{\partial p_i} + \frac{\partial X}{\partial p_i}\frac{\partial H}{\partial q_i} \right) \\
  \frac{dX}{dt}
      & = \sum_i \left( \frac{\partial X}{\partial q_i}\frac{\partial H}{\partial p_i} + \frac{\partial X}{\partial p_i}\frac{\partial H}{\partial q_i} \right) + \frac{\partial X}{\partial t} \\
      & = \left\{ X , H \right\} + \frac{\partial X}{\partial t} \\
\end{align}

\begin{align}
  \dot{q_i} & = \left\{ q_i , H \right\} \\
  \dot{p_i} & = \left\{ p_i , H \right\} \\
\end{align}

\section{Canonical transformation (正準変換)}


\subsection{Liouville's theorem}
\begin{equation}
  \int d\bf{Q} d\bf{P} = \int J d\bf{q} d\bf{p}
\end{equation}

where the Jacobian is the determinant of the matrix of partial derivatives, which we write as
\begin{equation}
  J \equiv \frac{\partial \left( \bf{Q} , \bf{P} \right)}{\partial \left( \bf{q} , \bf{p} \right)}
\end{equation}




% ============================================================================
%
% 電磁気学
%
% ============================================================================
\chapter{電磁気学}
\section{electrostatic field (静電場)}


\subsection{Coulomb's law (クーロンの法則)}
\begin{align}
  {\bf F}
      & = k \frac{q_1 \cdot q_2}{r^2} \\
      & = \frac{1}{4\pi \varepsilon_0 } \cdot \frac{q_1 \cdot q_2}{r^2}
\end{align}

\begin{align}
  {\bf F} ( {\bf r} )
      & = \frac{q}{4\pi \varepsilon_0} \sum_{i} \frac{Q_i ( {\bf r}_q - {\bf r}_i )}{ {| {\bf r}_q - {\bf r}_i |}^3 } \\
      & = q{\bf E} ({\bf r})
\end{align}


\subsubsection{真空の誘電率}
\begin{align}
  \varepsilon_0
      & = \frac{10^7}{4\pi c^2} \\
      & = 8.854\times 10^{-12} [C^2/N\cdot m^2]
\end{align}


\subsection{静電場}
\begin{align}
  {\bf E} ( {\bf r} )
      & = \frac{1}{4\pi \varepsilon_0} \sum_{i} \frac{Q_i ( {\bf r}_q - {\bf r}_i )}{ {| {\bf r}_q - {\bf r}_i |}^3 } \\
      & = \frac{1}{4\pi \varepsilon_0} \int_V \frac{\rho ({\bf r'}) ( {\bf r} - {\bf r'} )}{ {| {\bf r} - {\bf r'} |}^3 } dV'
\end{align}

原点に$q$がある場合
\begin{equation}
  {\bf E} ({\bf r}) = \frac{q}{4\pi \varepsilon_0} {\bf e}_r
\end{equation}

\begin{equation}
  {\bf F}( {\bf r}) = q {\bf E} ( {\bf r})
\end{equation}


\subsection{ガウスの法則}
\[ \int_{S_0} {\bf E}_n \cdot d{\bf S} = \frac{Q}{\varepsilon_0} \]
\[ \div {\bf E} = \frac{Q}{\varepsilon_0} \]


\subsection{静電ポテンシャル}
\[ dW = {\bf F} \cdot d{\bf s} = q{\bf E}\cdot d{\bf S} \]
\[ W_{A\rightarrow B} = q \int_{A}^{B} {\bf E} \cdot d{\bf s} = q \left[ \phi ({\bf r}_A) - \phi ({\bf r}_B) \right] \]


\subsubsection{基準点付きの静電ポテンシャル}
\[ \phi ({\bf r}) = - \int_{{\bf r}_0}^{{\bf r}} {\bf E} \cdot d{\bf s} \]


\subsubsection{電場との関係}
\begin{equation}
  {\bf E} ({\bf r}) \cdot d{\bf s} = -d \phi ({\bf r})
\end{equation}

\begin{equation}
  {\bf E} ({\bf r}) = - \grad \phi ({\bf r})
\end{equation}

\section{Magnetostatics (静磁気学)}


\subsection{Ampere's force law}
\[ F_m = k_m \frac{ I_1 I_2 }{r} = \frac{ {\mu}_0 }{ 2\pi  } \frac{ I_1 I_2 }{ r } \]
\[ \bf{F}_{12} = \frac{\mu_0}{4 \pi} I_1 I_2 \oint_{C_1} \oint_{C_2} \frac {d \bf{s_2} \times (d \bf{s_1} \times \hat{\bf{r}}_{12} )} {r_{12}^2} \]

\subsection{Biot-Savart Law}
\[ d\bf{B} = \frac{\mu_0}{4\pi} \frac{ I d\bf{l} \times \bf{r} }{ r^3 }  \]

\subsection{Ampere's circuital law}
\[ \oint_C \bf{B} \cdot \rm{d} \bf{ l } = {\mu}_0 \int\!\!\!\int_S \bf{J_f} \cdot \rm{d}\bf{S} = {\mu}_0 \bf{ I }_{enc} \]
\[ \rm{rot} \bf{B} = \nabla \times \bf{B} = {\mu} \bf{J}_f \]

\subsection{Lorentz force}
\[ \bf{F}_{mag} = q \left( \bf{v} \times \bf{B} \right)  \]
\[ \bf{F} = q \left( \bf{E} + \bf{v} \times \bf{B} \right)  \]
\section{Maxwell's equations (マクスウェルの方程式)}


\subsection{Integral form}
Gauss's law
\[ \oint_{S} \bf{D} \cdot \rm{d} \bf{A} = Q_{f,S} \]
Gauss's law for magnetism
\[ \oint_{S} \bf{B} \cdot \rm{d} \bf{A} = 0 \]
Maxwell-Faraday equation
\[ \oint_{\partial S} \bf{E} \cdot \rm{d} \bf{l} = - \frac {\partial \Phi_{B,S}}{\partial t} \]
Ampere's circuital law
\[ \oint_{\partial S} \bf{H} \cdot \rm{d} \bf{l} = I_{f,S} + \frac {\partial \Phi_{D,S}}{\partial t} \]

\subsection{Differential form}
Gauss's law
\[ \rm{div} \bf{D} = \nabla \cdot \bf{D} = {\rho}_{f} \]
Gauss's law for magnetism
\[ \rm{div} \bf{B} = \nabla \cdot \bf{B} = 0 \]
Maxwell-Faraday equation
\[ \rm{rot} \bf{E} = \nabla \times \bf{E} = - \frac{ \partial \bf{B} }{ \partial t} \]
Ampere's circuital law
\[ \rm{rot} \bf{H} = \nabla \times \bf{H} = \bf{J}_{f} + \frac{ \partial \bf{D} }{ \partial t} \]

\subsection{Definitions and units}
\[ \bf{D} = {\varepsilon}_{0} \bf{E} + \bf{P} = {\epsilon}_{0} \bf{E} \]
\[ \bf{B} = {\mu}_{0} \left( \bf{H} + \bf{M} \right) \]
$\bf{D}$ is the electric flux density. \ \ $ \bf{E} $ is the electric field. $ \bf{P} $ is the polarization density.\\
$\bf{B}$ is the magnetic flux density. \ \ $ \bf{H} $ is the magnetic field. $ \bf{M} $ is the magnetization density.\\
$ {\varepsilon}_{0} \approx  5.526\ 35\ \ldots \times 10^{7} [ \bf{A}^{2} s^{4} / kg \  m^{3} ] $ is the permittivity of free space.\\
$ {\mu}_{0} := 4\pi\times 10^{-7} [ \bf{N} / \bf{A}^{2} ] $ is the permeability of free space.\\

\subsection{Electromagnetic wave equation}

\[ \left( {\nabla}^2 - \frac{ 1 }{ c^2 } \frac{ \partial }{ \partial t} \right) \bf{E} = \Box \bf{E} = 0 \]
\[ \left( {\nabla}^2 - \frac{ 1 }{ c^2 } \frac{ \partial }{ \partial t} \right) \bf{B} = \Box \bf{B} = 0 \]
$ c = c_0 = { \frac{ 1 }{ \sqrt{ \mu_0 \varepsilon_0 } } } = 2.99792458 \times 10^8 [ m / s ] $ is the speed of light in vacuum.\\
\[ \Box = g^{\mu\nu} \partial_\nu \partial_\mu = \left( {\nabla}^2 - \frac{ 1 }{ c^2 } \frac{ \partial }{ \partial t} \right) \]
$ \Box $ is the d'Alembertian operator.
\section{(電磁ポテンシャル)}
\subsection{magnetic vector potential}
\[ {\bf B} = \nabla \times {\bf A} \]

\subsection{electric potential}
\[ {\bf E} = -\nabla \phi -\frac{ \partial {\bf A} }{ \partial t } \]

\subsection{Gauge freedom}
\begin{eqnarray*}
{\bf A'}& =   & {\bf A} + \nabla \chi  \\
\phi '& =   & \phi - \frac{ \partial \chi }{ \partial t } \\
\end{eqnarray*}

\subsection{Lorenz gauge condition}
\[ \nabla \cdot {\bf A} +\frac{ 1 }{ c^2 } \frac{ \partial \phi }{ \partial t } \]

\section{Electromagnetic four-potential}
\[ A^i = \left( \frac{ \phi }{ c } , {\bf A} \right) \]

\subsection{Lorenz gauge condition}
\[ \partial_i A^i = 0 \]

\subsection{Maxwell's equations}
\[  \]




% ============================================================================
%
% 熱力学
%
% ============================================================================
\chapter{熱力学}
% ============================================================================
%
% 物理学まとめノート
%
%
%
%
% ============================================================================
\section{Thermodynamic potential}
\subsection{Internal energy\\(内部エネルギー)}



\subsection{Helmholtz free energy\\(ヘルムホルツの自由エネルギー)}
\subsubsection{Definition}
\[ F \equiv U - TS \]
\subsubsection{Mathematical development}
\[ dF = -SdT -pdV \]
\[ S = -{\left( \frac{ \partial F }{ \partial T } \right)}_V \ ,\  p=-{\left( \frac{ \partial F }{ \partial V } \right)}_T \]
\subsubsection{}
\subsubsection{}


\subsection{Enthalpy(エンタルピー)}
\subsubsection{Definition}
\[ H \equiv U + pV \]
\subsubsection{Mathematical development}
\[ dH = TdS + Vdp \]
\[ T = {\left( \frac{ \partial H }{ \partial S } \right)}_p \ ,\  V={\left( \frac{ \partial H }{ \partial p } \right)}_S \]



\subsection{Gibbs free energy(ギブスの自由エネルギー)}

\subsection{Maxwell relations(マクスウェルの関係式)}





% ============================================================================
%
% 統計力学
%
% ============================================================================
\chapter{統計力学}
% ============================================================================
%
% 物理学まとめノート
%
%
%
%
% ============================================================================
\section{Mathematical formula}
\subsection{Gaussian integral}
\[ \int_{-\infty }^{\infty } e^{-x^2} dx = \sqrt[  ]{\mathstrut \pi } \]
\[ \int_{-\infty }^{\infty } e^{-ax^2} dx = \sqrt[  ]{\mathstrut \frac{ \pi }{ a } } \]
\[ \int_{-\infty }^{\infty } e^{-ax^b} dx = \frac{1}{b} a^{-1/b} \Gamma \left( \frac{1}{b} \right) = a^{-1/b} \Gamma \left( 1 + \frac{1}{b} \right) \]

\subsection{Stirling's approximation}
The formula as typically used in applications is
\[ \ln n! = n\ln n - n + O\left( \ln \left( n \right) \right) \]
often written
\[ n! \sim \sqrt[  ]{\mathstrut 2\pi n} {\left( \frac{n}{e} \right)}^n \]

\subsection{Gamma function}
\[ \Gamma \left( n \right) = \left( n -1 \right) ! \]
\[ \Gamma \left( \frac{1}{2} \right) = \sqrt[  ]{\mathstrut \pi } \]
for non-negative integer values of $n$ we have:
\[ \Gamma \left( \frac{1}{2} +n \right) = \frac{ \left( 2n \right) !}{ 4^n n! }\sqrt[  ]{\mathstrut \pi } = \frac{ \left( 2n-1 \right) !! }{ 2^2 } \sqrt[  ]{\mathstrut \pi } = \frac{\left( 2n \right) !}{2^{2n} n!} \sqrt[  ]{\mathstrut \pi }\]


% ============================================================================
%
% 物理学まとめノート
%
%
%
%
% ============================================================================
\section{Microcanonical ensemble (ミクロカノニカル分布)}


% ============================================================================
%
% 物理学まとめノート
%
%
%
%
% ============================================================================
\section{Canonical ensemble (カノニカル分布)}

\subsection{Boltzmann distribution(ボルツマン分布)}
\[ P \left( E_i \right) = \frac{1}{Z} \exp \left( -\beta E_i \right) \]
where $\beta $ is given by
\[ \beta = \frac{1}{k_B T} \]
$Z$ is the Partition Function, $k_B$ is Boltzmann's Constant, $T$ is temperature and $E_i$ is the energy of state $i$.

\subsection{Boltzmann constant(ボルツマン定数)}
\[ k_B = \frac{R}{N_A} = 1.3806504(24) \times 10^23   [J/K] \]
$R$ is the gas constant, $N_A$ is the Avogadro constant.

\subsection{Partition function(分配関数)}
\[ Z = \sum_i e^{-\beta E_i} \]


% ============================================================================
%
% 物理学まとめノート
%
%
%
%
% ============================================================================
\section{Grand canonical ensemble (グランドカノニカル分布)}

\subsection{基本的性質}
\[ \langle N \rangle = \sum_{i} N_i p_i = \frac{1}{\Xi \left( \beta ,\mu  \right)} \sum_i N_i e^{-\beta \left( E_i -\mu N_i \right)} = \frac{1}{\beta } \frac{\partial }{\partial \mu } \ln \Xi \left( \beta ,\mu  \right) \]





% ============================================================================
%
% 量子力学
%
% ============================================================================
\chapter{量子力学}
% ============================================================================
%
% 物理学まとめノート
%
%
%
%
% ============================================================================
\section{Old quantum theory}


% ============================================================================
%
% 物理学まとめノート
%
%
%
%
% ============================================================================
\section{Heisenberg picture (ハイゼンベルグ描像)}


\subsection{Heisenberg equation (ハイゼンベルグ方程式)}



\input{Physics/QuantumMechanics/SchrodingerPicture.tex}
% ============================================================================
%
% 物理学まとめノート
%
%
%
%
% ============================================================================
\section{Perturbation theory (摂動論)}


\subsection{時間依・存縮退のない摂動}

波動関数$ \ket{ \psi_n } $、$ E_n $を
\begin{eqnarray*}
\ket{ \psi_n }  & = & \ket{ \psi_n^{(0)} } + \lambda \ket{ \psi_n^{(1)} } + \lambda^2 \ket{ \psi_n^{(2)} } + \lambda^3 \ket{ \psi_n^{(3)} } + \cdots\\
E_n         & = & E_n^{(0)} + \lambda E_n^{(1)} + \lambda^2 E_n^{(2)} + \lambda^3 E_n^{(3)} + \cdots\\
\end{eqnarray*}
とし、波動方程式を
\[ \left( \hat{H} + \lambda \hat{H'} \right) \ket{ \psi_n } = E_n \ket{\psi_n} \]
とする。\\
1次の摂動項
\[ \ket{\psi_n^{(1)}} = \sum_m C_m \ket{\psi_m^{(0)}} \]
と置くと$ k \not= n $で、
\begin{eqnarray*}
C_k & = & -\frac{ \Braket{ \psi_k^{(0)} | \hat{H'} | \psi_n^{(0)} } }{ E_k^{(0)} - E_n^{(0)} } \\
C_n & = & 0 \\
\end{eqnarray*}
となり、
\begin{eqnarray*}
E_n^{(1)}   & = & \Braket{ \psi_n^{(0)} | \hat{H'} | \psi_n^{(0)} } \\
\ket{ \psi_n^{(1)}} & = & -\sum_{m\not= n} \frac{ \Braket{ \psi_m^{(0)} | \hat{H'} | \psi_n^{(0)} }}{ E_m^{(0)} - E_n^{(0)} } \ket{ \psi_m^{(0)} } \\
\end{eqnarray*}
である。
2次の摂動項
\[ E_n^{(2)} = -\sum_{m \not= n}\frac{ {\left| \Braket{\psi_m^{(0)} | \hat{H'} | \psi_n^{(0)} } \right|}^2 }{ E_m^{(0)} - E_n^{(0)} } \]
また摂動はおおよそ以下の条件下で近似として機能する
\[ \left| \lambda \Braket{ \psi_m^{(0)} | \hat{H'} | \psi_n^{(0)} } \right| \ll \left| E_m^{(0)} - E_n^{(0)} \right| \]

% ============================================================================
%
% 物理学まとめノート
%
%
%
%
% ============================================================================
\section{Dirac equation}

\subsection{Gamma matrices}
\subsubsection{}
%\begin{eqnarray}
\[
\gamma^0 =\left(
\begin{array}{cccc}
1& 0 & 0 & 0 \\
0& 1 & 0 & 0 \\
0& 0 & -1 & 0 \\
0& 0 & 0 & -1 \\
\end{array}
\right)
,
\gamma^1 =\left(
\begin{array}{cccc}
0& 0 & 0 & 1 \\
0& 0 & 1 & 0 \\
0& -1 & 0 & 0 \\
-1& 0 & 0 & 0 \\
\end{array}
\right)
%\end{eqnarray}
\]

%\begin{eqnarray}
\[
\gamma^2 =\left(
\begin{array}{cccc}
0& 0 & 0 & -i \\
0& 0 & i & 0 \\
0& i & 0 & 0 \\
-i& 0 & 0 & 0 \\
\end{array}
\right)
,
\gamma^3 =\left(
\begin{array}{cccc}
0& 0 & 1 & 0 \\
0& 0 & 0 & -1 \\
-1& 0 & 0 & 0 \\
0& 1 & 0 & 0 \\
\end{array}
\right)
%\end{eqnarray}
\]

\subsubsection{}
$ \eta^{\mu \nu } = ( +---) $
\[ \left[ \gamma^\mu , \gamma^\nu \right] \]




% ============================================================================
%
%
% 相対性理論
%
%
% ============================================================================
\chapter{相対性理論}
\section{特殊相対性理論}
Zur Elektrodynamik bewegter K\"{o}rper


% ============================================================================
%
% ローレンツ変換
%
% ============================================================================
\subsection{ローレンツ変換}
\begin{align}
  t' & = \frac{t - (v/c^2)x}{\sqrt[]{\mathstrut 1 - {(v/c)}^2}} \\
  x' & = \frac{x - vt}{\sqrt[]{\mathstrut 1 - {(v/c)}^2}} \\
  y' & = y \\
  z' & = z
\end{align}

\begin{equation}
  \left[
    \begin{array}{c}
      ct' \\ x' \\ y' \\ z' \\
    \end{array}
  \right]
  =
  \left[
    \begin{array}{cccc}
      \gamma & -\gamma \frac{v}{c} & 0 & 0 \\
      -\gamma \frac{v}{c} & \gamma & 0 & 0 \\
      0 & 0 & 0 & 0 \\
      0 & 0 & 0 & 0 \\
    \end{array}
  \right]
  \left[
    \begin{array}{c}
      ct \\ x \\ y \\ z \\
    \end{array}
  \right]
\end{equation}



% ============================================================================
%
%
% 相対論的電磁気学
%
%
% ============================================================================
\section{Electromagnetism in 4D(相対論的電磁気学)}


% ============================================================================
%
% ポテンシャル
%
% ============================================================================
\subsection{4-vector potential(4次元ポテンシャルベクトル)}
\begin{align}
  A^{\mu } & = \left( \phi /c , \bf{A} \right)\\
  A_{\mu } & = {\eta }_{\mu \nu } A^{\nu } = \left( - \phi /c , \bf{A} \right)
\end{align}

\subsection{electromagnetic field tensor}
\begin{align}
  F_{\mu \nu }
    & = {\partial}_{\mu } A_{\nu } - {\partial }_{\nu } A_{\mu } \\
    & = \left(
      \begin{array}{cccc}
        0 & -E_x / c & -E_y / c & -E_z / c \\
        E_x / c & 0 & B_z & -B_y \\
        E_y / c & -B_z & 0 & B_x \\
        E_z / c & B_y & -B_x & 0 \\
      \end{array}
    \right)
\end{align}
  \[ F_{\mu \nu }
      = {\partial}_{\mu } A_{\nu } - {\partial }_{\nu } A_{\mu }
      =
        \left(
        \begin{array}{cccc}
          0 & -E_x / c & -E_y / c & -E_z / c \\
          E_x / c & 0 & B_z & -B_y \\
          E_y / c & -B_z & 0 & B_x \\
          E_z / c & B_y & -B_x & 0 \\
        \end{array}
        \right) \]
  \subsubsection{Properties}
    \[ F_{\mu \nu} = - F_{\nu \mu} \]

\subsection{Four-current(4元電流)}
\begin{equation}
  j^\mu = \left( c\rho, \bf{j} \right)
\end{equation}

\subsubsection{Continuity equation}
\begin{equation}
  \partial_\mu j^\mu = 0
\end{equation}

\subsection{Maxwell's equations(マクスウェルの方程式)}
\subsubsection{Ampère's circuital law (with Maxwell's correction)}
\begin{align}
  {\bf \nabla} \times {\bf B}
    & = \mu_0 \left( {\bf J} + \varepsilon_0 \frac{\partial {\bf B}}{\partial t} \right) \\
  {\bf \nabla} \cdot {\bf E}
    & = \frac{\rho}{\varepsilon_0}
\end{align}
\begin{align}
  \partial_\mu F^{\mu \nu}
    & = -\mu_0 j^\mu \\
  \Box A^\mu - \partial^\mu \partial_\nu A^\nu
    & = -\mu_0 j^\mu
\end{align}

\subsubsection{Maxwell–Faraday equation (Faraday's law of induction)}
\begin{align}
  {\bf \nabla} \times {\bf E}
    & = - \frac{\partial {\bf B}}{\partial t} \\
  {\bf \nabla} \cdot {\bf B}
    & = 0
\end{align}
\begin{equation}
  \partial_\lambda F_{\mu \nu}
    + \partial_\mu F_{\nu \lambda}
    + \partial_\nu F_{\lambda \mu}
    = 0
\end{equation}



% ============================================================================
%
%
% エネルギー運動量テンソル
%
%
% ============================================================================
\section{Stress-energy tensor(エネルギー運動量テンソル)}


% ============================================================================
%
% エネルギー運動量テンソルの例
%
% ============================================================================
\subsection{Stress-energy in special situations}
\subsubsection{Stress-energy of a fluid in equilibrium}
\begin{align}
  T^{\alpha \beta }
    & = \left( \rho + \frac{p}{c^2} \right) u^{\alpha } u^{\beta } + p g^{\alpha \beta } \\
  g^{\alpha \beta }
    & = \left(
      \begin{array}{cccc}
        -{c}^{-2} & 0 & 0 & 0 \\
        0 & 1 & 0 & 0 \\
        0 & 0 & 1 & 0 \\
        0 & 0 & 0 & 1 \\
      \end{array}
    \right) \\
  T^{\alpha \beta }
    & = \left(
      \begin{array}{cccc}
        \rho & 0 & 0 & 0 \\
        0 & p & 0 & 0 \\
        0 & 0 & p & 0 \\
        0 & 0 & 0 & p \\
      \end{array}
    \right)
\end{align}


\subsubsection{Electromagnetic stress-energy tensor}
\begin{equation}
  T^{\mu \nu }
    = \frac{1}{{\mu }_0} \left( F^{\mu \alpha } g_{\alpha \beta } F^{\nu \beta }
        - \frac{1}{4} g^{\mu \nu } F_{\delta \gamma } F^{\delta \gamma } \right)
\end{equation}

% ============================================================================
%
%
% リーマン幾何学
%
%
% ============================================================================
\section{リーマン幾何学}


% ============================================================================
%
% 計量テンソル
%
% ============================================================================
\subsection{Metric tensor(計量テンソル)}
\begin{align}
  {ds}^2
    & = g_{ij}(x) {dx}^{i} {dx}^{j} = {g'}_{ij}(x') {dx'}^{i} {dx'}^{j} \\
  g^{ik} g_{kj}
    & = {{\delta }^i}_j \\
  g_{ij} \frac{\partial g^{ij}}{\partial x^k}
    & = -g^{ij} \frac{\partial g_{ij}}{\partial x^k} \\
  \frac{\partial g^{ij}}{\partial x^k}
    & = - g^{il} g^{jm} \frac{\partial g_{jm}}{\partial x^k} \\
  \frac{\partial g(x)}{\partial x^i}
    & = g(x) g^{jm} \frac{\partial g_{jm}}{\partial x^i} \\
  \frac{1}{\sqrt[]{\mathstrut -g}} \frac{\partial \sqrt[]{\mathstrut -g}}{\partial x^i}
    & = \frac{1}{2} g^{jk} \frac{\partial g_{jk}}{\partial x^i}
\end{align}


\subsubsection{Examples}
Flat spacetime with coordinates $ (t, x, y, z) $
\begin{equation}
  ds^2 = -dt^2 +dx^2 + dy^2 + dz^2
\end{equation}
In spherical coordinates $ (t, r, \theta, \phi) $
\begin{equation}
  ds^2 = -dt^2 + dr^2 + r^2d\Omega^2
\end{equation}
where
\begin{equation}
  d\Omega^2 = d\theta^2 + \sin^2 \theta d\phi^2
\end{equation}
is the standard metric on the 2-sphere.\\
\\
The round metric on a sphere
\begin{equation}
  ds^2 = d{\theta }^2 + sin^2\theta d\phi^2
\end{equation}


% ============================================================================
%
% クリストッフェル記号
%
% ============================================================================
\subsection{Christoffel symbols(クリストッフェル記号)}
\begin{align}
  {\Gamma }^{i}_{kl}
    & = \frac{1}{2} g^{im} \left( \frac{\partial g_{mk}}{\partial x^l} + \frac{\partial g_{ml}}{\partial x^k} - \frac{\partial g_{kl}}{\partial x^m} \right) \\
    & = \frac{1}{2} g^{im} \left( g_{mk,l} + g_{ml,k} - g_{kl,m} \right) \\
  {\Gamma }^{i}_{jk}
    & = {\Gamma }^{i}_{kj}
\end{align}


% ============================================================================
%
% 共変微分
%
% ============================================================================
\subsection{Covariant derivative(共変微分)}
\subsubsection{Examples}
\begin{align}
  {\nabla }_a \phi
    & = {\partial }_a \phi \\
  {\nabla }_b V^a
    & = {\partial }_b V^a + {\Gamma }^{a}_{cb} V^c \\
  {\nabla }_b V_a
    & = {\partial }_b V_a - {\Gamma }^{c}_{ab} V_c \\
  {\nabla }_c T^{ab}
    & = {\partial }_c T^{ab} + {\Gamma }^{a}_{dc} T^{db} + {\Gamma }^{b}_{dc} T^{ad} \\
  {\nabla }_c T_{ab}
    & = {\partial }_c T_{ab} - {\Gamma }^{d}_{ac} T_{db} - {\Gamma }^{d}_{bc} T_{ad}
\end{align}


% ============================================================================
%
% リーマンの曲率テンソル
%
% ============================================================================
\subsection{Riemann curvature tensor(リーマンの曲率テンソル)}
\[ {A_m}_{ [ ;i ;j ] }  \equiv A_{m ;i ;j } - A_{m ;j ;i } \equiv  \]
\[ {A_m}_{ [ ;i ;j ] }  = {R^b}_{ mij } A_b \]
\subsubsection{Symmetries and identities}
Skew symmetry
\[ R_{abcd} = -R_{bacd} = -R_{abdc} \]
Interchange symmetry
\[ R_{abcd} = R_{cdab} \]
First Bianchi identity
\[ R_{abcd} + R_{acdb} + R_{adbc} = 0 \]
\[ R_{a \left[ bcd \right] } = 0 \]
Second Bianchi identity
\[ R_{abcd;e} + R_{abde;c} + R_{abec;d} = \nabla_e R_{abcd} + \nabla_c R_{abde} + \nabla_d R_{abec} = 0 \]
\[ R_{ab \left[ cd;e \right] } = 0 \]

\subsubsection{aa}


% ============================================================================
%
% アインシュタインテンソル
%
% ============================================================================
\subsection{Einstein tensor(アインシュタインテンソル)}
\begin{equation}
  G_{\mu \nu} = R_{\mu \nu} - \frac{1}{2} g_{\mu \nu} R
\end{equation}
The Einstein tensor is symmetric
\begin{equation}
  G_{\mu \nu } = G_{\nu \mu }
\end{equation}
and, like the stress-energy tensor, divergenceless
\begin{equation}
  {G^{\mu \nu }}_{;\nu } = {\nabla }_{\nu } G^{\mu \nu } = 0
\end{equation}

% ============================================================================
%
% 一般相対性理論
%
%
%
%
% ============================================================================
\section{General relativity}



% ============================================================================
%
% 測地線方程式
%
% ============================================================================
\subsection{The geodesic equations}
\begin{equation}
  \frac{d^2 x^\alpha }{d\tau^2} + {\Gamma^\alpha }_{\beta \gamma } \frac{dx^\beta }{d\tau } \frac{dx^\gamma }{d\tau } = 0
\end{equation}


% ============================================================================
%
% アインシュタイン方程式
%
% ============================================================================
\subsection{Einstein field equations}
\[ R_{\mu \nu } - \frac{1}{2} g_{\mu \nu }R + g_{\mu \nu }\Lambda = \frac{8\pi G}{c^4} T_{\mu \nu } \]
\[ G_{\mu \nu } = \frac{8\pi G}{c^4} T_{\mu \nu } \]



% ============================================================================
%
% 3 + 1 分解
%
%
%
%
% ============================================================================
\section{3 + 1 Formalism}



% ============================================================================
%
% アインシュタイン方程式
%
% ============================================================================
\begin{align}
  \left( \frac{\partial}{\partial t} - {\mathcal L}_{\boldsymbol \beta} \right) K_{ij}
    & = -D_iD_j N + N \left\{ R_{ij} + KK_{ij} - 2K_{ik} K^k_j + 4\pi \left[ \left( S-E \right) \gamma_{ij} - 2S_{ij} \right] \right\} \\
  \left( \frac{\partial}{\partial t} - {\mathcal L}_{\boldsymbol \beta} \right) \gamma_{ij}
    & = -2NK_{ij}
\end{align}
\begin{gather}
  R + K^2 - K_{ij} K^{ij} = 16\pi E \\
  D_j K^j_i - D_i K = 8\pi p_i
\end{gather}




\end{document}
