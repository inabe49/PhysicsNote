% ============================================================================
%
% 物理学まとめノート
%
%
%
%
% ============================================================================
\section{Perturbation theory (摂動論)}


\subsection{時間依・存縮退のない摂動}

波動関数$ \ket{ \psi_n } $、$ E_n $を
\begin{eqnarray*}
\ket{ \psi_n }  & = & \ket{ \psi_n^{(0)} } + \lambda \ket{ \psi_n^{(1)} } + \lambda^2 \ket{ \psi_n^{(2)} } + \lambda^3 \ket{ \psi_n^{(3)} } + \cdots\\
E_n         & = & E_n^{(0)} + \lambda E_n^{(1)} + \lambda^2 E_n^{(2)} + \lambda^3 E_n^{(3)} + \cdots\\
\end{eqnarray*}
とし、波動方程式を
\[ \left( \hat{H} + \lambda \hat{H'} \right) \ket{ \psi_n } = E_n \ket{\psi_n} \]
とする。\\
1次の摂動項
\[ \ket{\psi_n^{(1)}} = \sum_m C_m \ket{\psi_m^{(0)}} \]
と置くと$ k \not= n $で、
\begin{eqnarray*}
C_k & = & -\frac{ \Braket{ \psi_k^{(0)} | \hat{H'} | \psi_n^{(0)} } }{ E_k^{(0)} - E_n^{(0)} } \\
C_n & = & 0 \\
\end{eqnarray*}
となり、
\begin{eqnarray*}
E_n^{(1)}   & = & \Braket{ \psi_n^{(0)} | \hat{H'} | \psi_n^{(0)} } \\
\ket{ \psi_n^{(1)}} & = & -\sum_{m\not= n} \frac{ \Braket{ \psi_m^{(0)} | \hat{H'} | \psi_n^{(0)} }}{ E_m^{(0)} - E_n^{(0)} } \ket{ \psi_m^{(0)} } \\
\end{eqnarray*}
である。
2次の摂動項
\[ E_n^{(2)} = -\sum_{m \not= n}\frac{ {\left| \Braket{\psi_m^{(0)} | \hat{H'} | \psi_n^{(0)} } \right|}^2 }{ E_m^{(0)} - E_n^{(0)} } \]
また摂動はおおよそ以下の条件下で近似として機能する
\[ \left| \lambda \Braket{ \psi_m^{(0)} | \hat{H'} | \psi_n^{(0)} } \right| \ll \left| E_m^{(0)} - E_n^{(0)} \right| \]
