% ============================================================================
%
% 物理学まとめノート
%
%
%
%
% ============================================================================
\section{Mathematical formula}
\subsection{Gaussian integral}
\[ \int_{-\infty }^{\infty } e^{-x^2} dx = \sqrt[  ]{\mathstrut \pi } \]
\[ \int_{-\infty }^{\infty } e^{-ax^2} dx = \sqrt[  ]{\mathstrut \frac{ \pi }{ a } } \]
\[ \int_{-\infty }^{\infty } e^{-ax^b} dx = \frac{1}{b} a^{-1/b} \Gamma \left( \frac{1}{b} \right) = a^{-1/b} \Gamma \left( 1 + \frac{1}{b} \right) \]

\subsection{Stirling's approximation}
The formula as typically used in applications is
\[ \ln n! = n\ln n - n + O\left( \ln \left( n \right) \right) \]
often written
\[ n! \sim \sqrt[  ]{\mathstrut 2\pi n} {\left( \frac{n}{e} \right)}^n \]

\subsection{Gamma function}
\[ \Gamma \left( n \right) = \left( n -1 \right) ! \]
\[ \Gamma \left( \frac{1}{2} \right) = \sqrt[  ]{\mathstrut \pi } \]
for non-negative integer values of $n$ we have:
\[ \Gamma \left( \frac{1}{2} +n \right) = \frac{ \left( 2n \right) !}{ 4^n n! }\sqrt[  ]{\mathstrut \pi } = \frac{ \left( 2n-1 \right) !! }{ 2^2 } \sqrt[  ]{\mathstrut \pi } = \frac{\left( 2n \right) !}{2^{2n} n!} \sqrt[  ]{\mathstrut \pi }\]

