% ============================================================================
%
%
% リーマン幾何学
%
%
% ============================================================================
\section{リーマン幾何学}


% ============================================================================
%
% 計量テンソル
%
% ============================================================================
\subsection{Metric tensor(計量テンソル)}
\begin{align}
  {ds}^2
    & = g_{ij}(x) {dx}^{i} {dx}^{j} = {g'}_{ij}(x') {dx'}^{i} {dx'}^{j} \\
  g^{ik} g_{kj}
    & = {{\delta }^i}_j \\
  g_{ij} \frac{\partial g^{ij}}{\partial x^k}
    & = -g^{ij} \frac{\partial g_{ij}}{\partial x^k} \\
  \frac{\partial g^{ij}}{\partial x^k}
    & = - g^{il} g^{jm} \frac{\partial g_{jm}}{\partial x^k} \\
  \frac{\partial g(x)}{\partial x^i}
    & = g(x) g^{jm} \frac{\partial g_{jm}}{\partial x^i} \\
  \frac{1}{\sqrt[]{\mathstrut -g}} \frac{\partial \sqrt[]{\mathstrut -g}}{\partial x^i}
    & = \frac{1}{2} g^{jk} \frac{\partial g_{jk}}{\partial x^i}
\end{align}


\subsubsection{Examples}
Flat spacetime with coordinates $ (t, x, y, z) $
\begin{equation}
  ds^2 = -dt^2 +dx^2 + dy^2 + dz^2
\end{equation}
In spherical coordinates $ (t, r, \theta, \phi) $
\begin{equation}
  ds^2 = -dt^2 + dr^2 + r^2d\Omega^2
\end{equation}
where
\begin{equation}
  d\Omega^2 = d\theta^2 + \sin^2 \theta d\phi^2
\end{equation}
is the standard metric on the 2-sphere.\\
\\
The round metric on a sphere
\begin{equation}
  ds^2 = d{\theta }^2 + sin^2\theta d\phi^2
\end{equation}


% ============================================================================
%
% クリストッフェル記号
%
% ============================================================================
\subsection{Christoffel symbols(クリストッフェル記号)}
\begin{align}
  {\Gamma }^{i}_{kl}
    & = \frac{1}{2} g^{im} \left( \frac{\partial g_{mk}}{\partial x^l} + \frac{\partial g_{ml}}{\partial x^k} - \frac{\partial g_{kl}}{\partial x^m} \right) \\
    & = \frac{1}{2} g^{im} \left( g_{mk,l} + g_{ml,k} - g_{kl,m} \right) \\
  {\Gamma }^{i}_{jk}
    & = {\Gamma }^{i}_{kj}
\end{align}


% ============================================================================
%
% 共変微分
%
% ============================================================================
\subsection{Covariant derivative(共変微分)}
\subsubsection{Examples}
\begin{align}
  {\nabla }_a \phi
    & = {\partial }_a \phi \\
  {\nabla }_b V^a
    & = {\partial }_b V^a + {\Gamma }^{a}_{cb} V^c \\
  {\nabla }_b V_a
    & = {\partial }_b V_a - {\Gamma }^{c}_{ab} V_c \\
  {\nabla }_c T^{ab}
    & = {\partial }_c T^{ab} + {\Gamma }^{a}_{dc} T^{db} + {\Gamma }^{b}_{dc} T^{ad} \\
  {\nabla }_c T_{ab}
    & = {\partial }_c T_{ab} - {\Gamma }^{d}_{ac} T_{db} - {\Gamma }^{d}_{bc} T_{ad}
\end{align}


% ============================================================================
%
% リーマンの曲率テンソル
%
% ============================================================================
\subsection{Riemann curvature tensor(リーマンの曲率テンソル)}
\[ {A_m}_{ [ ;i ;j ] }  \equiv A_{m ;i ;j } - A_{m ;j ;i } \equiv  \]
\[ {A_m}_{ [ ;i ;j ] }  = {R^b}_{ mij } A_b \]
\subsubsection{Symmetries and identities}
Skew symmetry
\[ R_{abcd} = -R_{bacd} = -R_{abdc} \]
Interchange symmetry
\[ R_{abcd} = R_{cdab} \]
First Bianchi identity
\[ R_{abcd} + R_{acdb} + R_{adbc} = 0 \]
\[ R_{a \left[ bcd \right] } = 0 \]
Second Bianchi identity
\[ R_{abcd;e} + R_{abde;c} + R_{abec;d} = \nabla_e R_{abcd} + \nabla_c R_{abde} + \nabla_d R_{abec} = 0 \]
\[ R_{ab \left[ cd;e \right] } = 0 \]

\subsubsection{aa}


% ============================================================================
%
% アインシュタインテンソル
%
% ============================================================================
\subsection{Einstein tensor(アインシュタインテンソル)}
\begin{equation}
  G_{\mu \nu} = R_{\mu \nu} - \frac{1}{2} g_{\mu \nu} R
\end{equation}
The Einstein tensor is symmetric
\begin{equation}
  G_{\mu \nu } = G_{\nu \mu }
\end{equation}
and, like the stress-energy tensor, divergenceless
\begin{equation}
  {G^{\mu \nu }}_{;\nu } = {\nabla }_{\nu } G^{\mu \nu } = 0
\end{equation}
