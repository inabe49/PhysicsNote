\section{特殊相対性理論}
Zur Elektrodynamik bewegter K\"{o}rper


% ============================================================================
%
% ローレンツ変換
%
% ============================================================================
\subsection{ローレンツ変換}
\begin{align}
  t' & = \frac{t - (v/c^2)x}{\sqrt[]{\mathstrut 1 - {(v/c)}^2}} \\
  x' & = \frac{x - vt}{\sqrt[]{\mathstrut 1 - {(v/c)}^2}} \\
  y' & = y \\
  z' & = z
\end{align}

\begin{equation}
  \left[
    \begin{array}{c}
      ct' \\ x' \\ y' \\ z' \\
    \end{array}
  \right]
  =
  \left[
    \begin{array}{cccc}
      \gamma & -\gamma \frac{v}{c} & 0 & 0 \\
      -\gamma \frac{v}{c} & \gamma & 0 & 0 \\
      0 & 0 & 0 & 0 \\
      0 & 0 & 0 & 0 \\
    \end{array}
  \right]
  \left[
    \begin{array}{c}
      ct \\ x \\ y \\ z \\
    \end{array}
  \right]
\end{equation}



% ============================================================================
%
%
% 相対論的電磁気学
%
%
% ============================================================================
\section{Electromagnetism in 4D(相対論的電磁気学)}


% ============================================================================
%
% ポテンシャル
%
% ============================================================================
\subsection{4-vector potential(4次元ポテンシャルベクトル)}
\begin{align}
  A^{\mu } & = \left( \phi /c , \bf{A} \right)\\
  A_{\mu } & = {\eta }_{\mu \nu } A^{\nu } = \left( - \phi /c , \bf{A} \right)
\end{align}

\subsection{electromagnetic field tensor}
\begin{align}
  F_{\mu \nu }
    & = {\partial}_{\mu } A_{\nu } - {\partial }_{\nu } A_{\mu } \\
    & = \left(
      \begin{array}{cccc}
        0 & -E_x / c & -E_y / c & -E_z / c \\
        E_x / c & 0 & B_z & -B_y \\
        E_y / c & -B_z & 0 & B_x \\
        E_z / c & B_y & -B_x & 0 \\
      \end{array}
    \right)
\end{align}
  \[ F_{\mu \nu }
      = {\partial}_{\mu } A_{\nu } - {\partial }_{\nu } A_{\mu }
      =
        \left(
        \begin{array}{cccc}
          0 & -E_x / c & -E_y / c & -E_z / c \\
          E_x / c & 0 & B_z & -B_y \\
          E_y / c & -B_z & 0 & B_x \\
          E_z / c & B_y & -B_x & 0 \\
        \end{array}
        \right) \]
  \subsubsection{Properties}
    \[ F_{\mu \nu} = - F_{\nu \mu} \]

\subsection{Four-current(4元電流)}
\begin{equation}
  j^\mu = \left( c\rho, \bf{j} \right)
\end{equation}

\subsubsection{Continuity equation}
\begin{equation}
  \partial_\mu j^\mu = 0
\end{equation}

\subsection{Maxwell's equations(マクスウェルの方程式)}
\subsubsection{Ampère's circuital law (with Maxwell's correction)}
\begin{align}
  {\bf \nabla} \times {\bf B}
    & = \mu_0 \left( {\bf J} + \varepsilon_0 \frac{\partial {\bf B}}{\partial t} \right) \\
  {\bf \nabla} \cdot {\bf E}
    & = \frac{\rho}{\varepsilon_0}
\end{align}
\begin{align}
  \partial_\mu F^{\mu \nu}
    & = -\mu_0 j^\mu \\
  \Box A^\mu - \partial^\mu \partial_\nu A^\nu
    & = -\mu_0 j^\mu
\end{align}

\subsubsection{Maxwell–Faraday equation (Faraday's law of induction)}
\begin{align}
  {\bf \nabla} \times {\bf E}
    & = - \frac{\partial {\bf B}}{\partial t} \\
  {\bf \nabla} \cdot {\bf B}
    & = 0
\end{align}
\begin{equation}
  \partial_\lambda F_{\mu \nu}
    + \partial_\mu F_{\nu \lambda}
    + \partial_\nu F_{\lambda \mu}
    = 0
\end{equation}



% ============================================================================
%
%
% エネルギー運動量テンソル
%
%
% ============================================================================
\section{Stress-energy tensor(エネルギー運動量テンソル)}


% ============================================================================
%
% エネルギー運動量テンソルの例
%
% ============================================================================
\subsection{Stress-energy in special situations}
\subsubsection{Stress-energy of a fluid in equilibrium}
\begin{align}
  T^{\alpha \beta }
    & = \left( \rho + \frac{p}{c^2} \right) u^{\alpha } u^{\beta } + p g^{\alpha \beta } \\
  g^{\alpha \beta }
    & = \left(
      \begin{array}{cccc}
        -{c}^{-2} & 0 & 0 & 0 \\
        0 & 1 & 0 & 0 \\
        0 & 0 & 1 & 0 \\
        0 & 0 & 0 & 1 \\
      \end{array}
    \right) \\
  T^{\alpha \beta }
    & = \left(
      \begin{array}{cccc}
        \rho & 0 & 0 & 0 \\
        0 & p & 0 & 0 \\
        0 & 0 & p & 0 \\
        0 & 0 & 0 & p \\
      \end{array}
    \right)
\end{align}


\subsubsection{Electromagnetic stress-energy tensor}
\begin{equation}
  T^{\mu \nu }
    = \frac{1}{{\mu }_0} \left( F^{\mu \alpha } g_{\alpha \beta } F^{\nu \beta }
        - \frac{1}{4} g^{\mu \nu } F_{\delta \gamma } F^{\delta \gamma } \right)
\end{equation}
