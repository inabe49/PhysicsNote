% ============================================================================
%
% 一般相対性理論
%
%
%
%
% ============================================================================
\section{General relativity}



% ============================================================================
%
% 測地線方程式
%
% ============================================================================
\subsection{The geodesic equations}
\begin{equation}
  \frac{d^2 x^\alpha }{d\tau^2} + {\Gamma^\alpha }_{\beta \gamma } \frac{dx^\beta }{d\tau } \frac{dx^\gamma }{d\tau } = 0
\end{equation}


% ============================================================================
%
% アインシュタイン方程式
%
% ============================================================================
\subsection{Einstein field equations (アインシュタイン方程式)}
\begin{equation}
  R_{\mu \nu } - \frac{1}{2} g_{\mu \nu }R + g_{\mu \nu }\Lambda = \frac{8\pi G}{c^4} T_{\mu \nu }
\end{equation}

\begin{equation}
  G_{\mu \nu } = \frac{8\pi G}{c^4} T_{\mu \nu }
\end{equation}



% ============================================================================
%
% 3 + 1 分解
%
%
%
%
% ============================================================================
\section{3 + 1 Formalism}



% ============================================================================
%
% アインシュタイン方程式
%
% ============================================================================
\begin{align}
  \left( \frac{\partial}{\partial t} - {\mathcal L}_{\boldsymbol \beta} \right) K_{ij}
    & = -D_iD_j N + N \left\{ R_{ij} + KK_{ij} - 2K_{ik} K^k_j + 4\pi \left[ \left( S-E \right) \gamma_{ij} - 2S_{ij} \right] \right\} \\
  \left( \frac{\partial}{\partial t} - {\mathcal L}_{\boldsymbol \beta} \right) \gamma_{ij}
    & = -2NK_{ij}
\end{align}
\begin{gather}
  R + K^2 - K_{ij} K^{ij} = 16\pi E \\
  D_j K^j_i - D_i K = 8\pi p_i
\end{gather}
