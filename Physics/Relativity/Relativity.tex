\section{特殊相対性理論}
Zur Elektrodynamik bewegter K\"{o}rper


% ============================================================================
%
% ローレンツ変換
%
% ============================================================================
\subsection{ローレンツ変換}
\begin{align}
  t' & = \frac{t - (v/c^2)x}{\sqrt[]{\mathstrut 1 - {(v/c)}^2}} \\
  x' & = \frac{x - vt}{\sqrt[]{\mathstrut 1 - {(v/c)}^2}} \\
  y' & = y \\
  z' & = z
\end{align}

\begin{equation}
  \left[
    \begin{array}{c}
      ct' \\ x' \\ y' \\ z' \\
    \end{array}
  \right]
  =
  \left[
    \begin{array}{cccc}
      \gamma & -\gamma \frac{v}{c} & 0 & 0 \\
      -\gamma \frac{v}{c} & \gamma & 0 & 0 \\
      0 & 0 & 0 & 0 \\
      0 & 0 & 0 & 0 \\
    \end{array}
  \right]
  \left[
    \begin{array}{c}
      ct \\ x \\ y \\ z \\
    \end{array}
  \right]
\end{equation}



% ============================================================================
%
%
% 相対論的電磁気学
%
%
% ============================================================================
\section{Electromagnetism in 4D(相対論的電磁気学)}


% ============================================================================
%
% ポテンシャル
%
% ============================================================================
\subsection{4-vector potential(4次元ポテンシャルベクトル)}
\begin{align}
  A^{\mu } & = \left( \phi /c , \bf{A} \right)\\
  A_{\mu } & = {\eta }_{\mu \nu } A^{\nu } = \left( - \phi /c , \bf{A} \right)
\end{align}

\subsection{electromagnetic field tensor}
\begin{align}
  F_{\mu \nu }
    & = {\partial}_{\mu } A_{\nu } - {\partial }_{\nu } A_{\mu } \\
    & = \left(
      \begin{array}{cccc}
        0 & -E_x / c & -E_y / c & -E_z / c \\
        E_x / c & 0 & B_z & -B_y \\
        E_y / c & -B_z & 0 & B_x \\
        E_z / c & B_y & -B_x & 0 \\
      \end{array}
    \right)
\end{align}
  \[ F_{\mu \nu }
      = {\partial}_{\mu } A_{\nu } - {\partial }_{\nu } A_{\mu }
      =
        \left(
        \begin{array}{cccc}
          0 & -E_x / c & -E_y / c & -E_z / c \\
          E_x / c & 0 & B_z & -B_y \\
          E_y / c & -B_z & 0 & B_x \\
          E_z / c & B_y & -B_x & 0 \\
        \end{array}
        \right) \]
  \subsubsection{Properties}
    \[ F_{\mu \nu} = - F_{\nu \mu} \]

\subsection{Four-current(4元電流)}
\begin{equation}
  j^\mu = \left( c\rho, \bf{j} \right)
\end{equation}

\subsubsection{Continuity equation}
\begin{equation}
  \partial_\mu j^\mu = 0
\end{equation}

\subsection{Maxwell's equations(マクスウェルの方程式)}
\subsubsection{Ampère's circuital law (with Maxwell's correction)}
\begin{align}
  {\bf \nabla} \times {\bf B}
    & = \mu_0 \left( {\bf J} + \varepsilon_0 \frac{\partial {\bf B}}{\partial t} \right) \\
  {\bf \nabla} \cdot {\bf E}
    & = \frac{\rho}{\varepsilon_0}
\end{align}
\begin{align}
  \partial_\mu F^{\mu \nu}
    & = -\mu_0 j^\mu \\
  \Box A^\mu - \partial^\mu \partial_\nu A^\nu
    & = -\mu_0 j^\mu
\end{align}

\subsubsection{Maxwell–Faraday equation (Faraday's law of induction)}
\begin{align}
  {\bf \nabla} \times {\bf E}
    & = - \frac{\partial {\bf B}}{\partial t} \\
  {\bf \nabla} \cdot {\bf B}
    & = 0
\end{align}
\begin{equation}
  \partial_\lambda F_{\mu \nu}
    + \partial_\mu F_{\nu \lambda}
    + \partial_\nu F_{\lambda \mu}
    = 0
\end{equation}



% ============================================================================
%
%
% エネルギー運動量テンソル
%
%
% ============================================================================
\section{Stress-energy tensor(エネルギー運動量テンソル)}


% ============================================================================
%
% エネルギー運動量テンソルの例
%
% ============================================================================
\subsection{Stress-energy in special situations}
\subsubsection{Stress-energy of a fluid in equilibrium}
\begin{align}
  T^{\alpha \beta }
    & = \left( \rho + \frac{p}{c^2} \right) u^{\alpha } u^{\beta } + p g^{\alpha \beta } \\
  g^{\alpha \beta }
    & = \left(
      \begin{array}{cccc}
        -{c}^{-2} & 0 & 0 & 0 \\
        0 & 1 & 0 & 0 \\
        0 & 0 & 1 & 0 \\
        0 & 0 & 0 & 1 \\
      \end{array}
    \right) \\
  T^{\alpha \beta }
    & = \left(
      \begin{array}{cccc}
        \rho & 0 & 0 & 0 \\
        0 & p & 0 & 0 \\
        0 & 0 & p & 0 \\
        0 & 0 & 0 & p \\
      \end{array}
    \right)
\end{align}


\subsubsection{Electromagnetic stress-energy tensor}
\begin{equation}
  T^{\mu \nu }
    = \frac{1}{{\mu }_0} \left( F^{\mu \alpha } g_{\alpha \beta } F^{\nu \beta }
        - \frac{1}{4} g^{\mu \nu } F_{\delta \gamma } F^{\delta \gamma } \right)
\end{equation}



% ============================================================================
%
%
% リーマン幾何学
%
%
% ============================================================================
\section{リーマン幾何学}


% ============================================================================
%
% 計量テンソル
%
% ============================================================================
\subsection{Metric tensor(計量テンソル)}
\begin{align}
  {ds}^2
    & = g_{ij}(x) {dx}^{i} {dx}^{j} = {g'}_{ij}(x') {dx'}^{i} {dx'}^{j} \\
  g^{ik} g_{kj}
    & = {{\delta }^i}_j \\
  g_{ij} \frac{\partial g^{ij}}{\partial x^k}
    & = -g^{ij} \frac{\partial g_{ij}}{\partial x^k} \\
  \frac{\partial g^{ij}}{\partial x^k}
    & = - g^{il} g^{jm} \frac{\partial g_{jm}}{\partial x^k} \\
  \frac{\partial g(x)}{\partial x^i}
    & = g(x) g^{jm} \frac{\partial g_{jm}}{\partial x^i} \\
  \frac{1}{\sqrt[]{\mathstrut -g}} \frac{\partial \sqrt[]{\mathstrut -g}}{\partial x^i}
    & = \frac{1}{2} g^{jk} \frac{\partial g_{jk}}{\partial x^i}
\end{align}


\subsubsection{Examples}
Flat spacetime with coordinates $ (t, x, y, z) $
\begin{equation}
  ds^2 = -dt^2 +dx^2 + dy^2 + dz^2
\end{equation}
In spherical coordinates $ (t, r, \theta, \phi) $
\begin{equation}
  ds^2 = -dt^2 + dr^2 + r^2d\Omega^2
\end{equation}
where
\begin{equation}
  d\Omega^2 = d\theta^2 + \sin^2 \theta d\phi^2
\end{equation}
is the standard metric on the 2-sphere.\\
\\
The round metric on a sphere
\begin{equation}
  ds^2 = d{\theta }^2 + sin^2\theta d\phi^2
\end{equation}


% ============================================================================
%
% クリストッフェル記号
%
% ============================================================================
\subsection{Christoffel symbols(クリストッフェル記号)}
\begin{align}
  {\Gamma }^{i}_{kl}
    & = \frac{1}{2} g^{im} \left( \frac{\partial g_{mk}}{\partial x^l} + \frac{\partial g_{ml}}{\partial x^k} - \frac{\partial g_{kl}}{\partial x^m} \right) \\
    & = \frac{1}{2} g^{im} \left( g_{mk,l} + g_{ml,k} - g_{kl,m} \right) \\
  {\Gamma }^{i}_{jk}
    & = {\Gamma }^{i}_{kj}
\end{align}


% ============================================================================
%
% 共変微分
%
% ============================================================================
\subsection{Covariant derivative(共変微分)}
\subsubsection{Examples}
\begin{align}
  {\nabla }_a \phi
    & = {\partial }_a \phi \\
  {\nabla }_b V^a
    & = {\partial }_b V^a + {\Gamma }^{a}_{cb} V^c \\
  {\nabla }_b V_a
    & = {\partial }_b V_a - {\Gamma }^{c}_{ab} V_c \\
  {\nabla }_c T^{ab}
    & = {\partial }_c T^{ab} + {\Gamma }^{a}_{dc} T^{db} + {\Gamma }^{b}_{dc} T^{ad} \\
  {\nabla }_c T_{ab}
    & = {\partial }_c T_{ab} - {\Gamma }^{d}_{ac} T_{db} - {\Gamma }^{d}_{bc} T_{ad}
\end{align}


% ============================================================================
%
% リーマンの曲率テンソル
%
% ============================================================================
\subsection{Riemann curvature tensor(リーマンの曲率テンソル)}
\[ {A_m}_{ [ ;i ;j ] }  \equiv A_{m ;i ;j } - A_{m ;j ;i } \equiv  \]
\[ {A_m}_{ [ ;i ;j ] }  = {R^b}_{ mij } A_b \]
\subsubsection{Symmetries and identities}
Skew symmetry
\[ R_{abcd} = -R_{bacd} = -R_{abdc} \]
Interchange symmetry
\[ R_{abcd} = R_{cdab} \]
First Bianchi identity
\[ R_{abcd} + R_{acdb} + R_{adbc} = 0 \]
\[ R_{a \left[ bcd \right] } = 0 \]
Second Bianchi identity
\[ R_{abcd;e} + R_{abde;c} + R_{abec;d} = \nabla_e R_{abcd} + \nabla_c R_{abde} + \nabla_d R_{abec} = 0 \]
\[ R_{ab \left[ cd;e \right] } = 0 \]

\subsubsection{aa}


% ============================================================================
%
% アインシュタインテンソル
%
% ============================================================================
\subsection{Einstein tensor(アインシュタインテンソル)}
\begin{equation}
  G_{\mu \nu} = R_{\mu \nu} - \frac{1}{2} g_{\mu \nu} R
\end{equation}
The Einstein tensor is symmetric
\begin{equation}
  G_{\mu \nu } = G_{\nu \mu }
\end{equation}
and, like the stress-energy tensor, divergenceless
\begin{equation}
  {G^{\mu \nu }}_{;\nu } = {\nabla }_{\nu } G^{\mu \nu } = 0
\end{equation}



% ============================================================================
%
% 一般相対性理論
%
%
%
%
% ============================================================================
\section{General relativity}



% ============================================================================
%
% 測地線方程式
%
% ============================================================================
\subsection{The geodesic equations}
\begin{equation}
  \frac{d^2 x^\alpha }{d\tau^2} + {\Gamma^\alpha }_{\beta \gamma } \frac{dx^\beta }{d\tau } \frac{dx^\gamma }{d\tau } = 0
\end{equation}


% ============================================================================
%
% アインシュタイン方程式
%
% ============================================================================
\subsection{Einstein field equations}
\[ R_{\mu \nu } - \frac{1}{2} g_{\mu \nu }R + g_{\mu \nu }\Lambda = \frac{8\pi G}{c^4} T_{\mu \nu } \]
\[ G_{\mu \nu } = \frac{8\pi G}{c^4} T_{\mu \nu } \]



% ============================================================================
%
% 3 + 1 分解
%
%
%
%
% ============================================================================
\section{3 + 1 Formalism}



% ============================================================================
%
% アインシュタイン方程式
%
% ============================================================================
\begin{align}
  \left( \frac{\partial}{\partial t} - {\mathcal L}_{\boldsymbol \beta} \right) K_{ij}
    & = -D_iD_j N + N \left\{ R_{ij} + KK_{ij} - 2K_{ik} K^k_j + 4\pi \left[ \left( S-E \right) \gamma_{ij} - 2S_{ij} \right] \right\} \\
  \left( \frac{\partial}{\partial t} - {\mathcal L}_{\boldsymbol \beta} \right) \gamma_{ij}
    & = -2NK_{ij}
\end{align}
\begin{gather}
  R + K^2 - K_{ij} K^{ij} = 16\pi E \\
  D_j K^j_i - D_i K = 8\pi p_i
\end{gather}
