\section{electrostatic field (静電場)}


\subsection{Coulomb's law (クーロンの法則)}
\begin{align}
  {\bf F}
      & = k \frac{q_1 \cdot q_2}{r^2} \\
      & = \frac{1}{4\pi \varepsilon_0 } \cdot \frac{q_1 \cdot q_2}{r^2}
\end{align}

\begin{align}
  {\bf F} ( {\bf r} )
      & = \frac{q}{4\pi \varepsilon_0} \sum_{i} \frac{Q_i ( {\bf r}_q - {\bf r}_i )}{ {| {\bf r}_q - {\bf r}_i |}^3 } \\
      & = q{\bf E} ({\bf r})
\end{align}


\subsubsection{真空の誘電率}
\begin{align}
  \varepsilon_0
      & = \frac{10^7}{4\pi c^2} \\
      & = 8.854\times 10^{-12} [C^2/N\cdot m^2]
\end{align}


\subsection{静電場}
\begin{align}
  {\bf E} ( {\bf r} )
      & = \frac{1}{4\pi \varepsilon_0} \sum_{i} \frac{Q_i ( {\bf r}_q - {\bf r}_i )}{ {| {\bf r}_q - {\bf r}_i |}^3 } \\
      & = \frac{1}{4\pi \varepsilon_0} \int_V \frac{\rho ({\bf r'}) ( {\bf r} - {\bf r'} )}{ {| {\bf r} - {\bf r'} |}^3 } dV'
\end{align}

原点に$q$がある場合
\begin{equation}
  {\bf E} ({\bf r}) = \frac{q}{4\pi \varepsilon_0} {\bf e}_r
\end{equation}

\begin{equation}
  {\bf F}( {\bf r}) = q {\bf E} ( {\bf r})
\end{equation}


\subsection{ガウスの法則}
\[ \int_{S_0} {\bf E}_n \cdot d{\bf S} = \frac{Q}{\varepsilon_0} \]
\[ \div {\bf E} = \frac{Q}{\varepsilon_0} \]


\subsection{静電ポテンシャル}
\[ dW = {\bf F} \cdot d{\bf s} = q{\bf E}\cdot d{\bf S} \]
\[ W_{A\rightarrow B} = q \int_{A}^{B} {\bf E} \cdot d{\bf s} = q \left[ \phi ({\bf r}_A) - \phi ({\bf r}_B) \right] \]


\subsubsection{基準点付きの静電ポテンシャル}
\[ \phi ({\bf r}) = - \int_{{\bf r}_0}^{{\bf r}} {\bf E} \cdot d{\bf s} \]


\subsubsection{電場との関係}
\begin{equation}
  {\bf E} ({\bf r}) \cdot d{\bf s} = -d \phi ({\bf r})
\end{equation}

\begin{equation}
  {\bf E} ({\bf r}) = - \grad \phi ({\bf r})
\end{equation}
